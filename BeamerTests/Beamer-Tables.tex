\input{mmd-beamer-header}
\def\mytitle{Test}
\def\latexmode{beamer}
\def\theme{keynote-gradient}
\input{mmd-beamer-begin-doc}

\begin{frame}

\frametitle{Tables}
\label{tables}

\begin{table}[htbp]
\begin{minipage}{\linewidth}
\setlength{\tymax}{0.5\linewidth}
\centering
\small
\begin{tabulary}{\textwidth}{@{}LCC@{}} \toprule
Features&MultiMarkdown&Crayons\\
\midrule
Melts in warm places&No&Yes\\
Mistakes can be easily fixed&Yes&No\\
Easy to copy documents for friends&Yes&No\\
Fun at parties&No&Why not?\\

\midrule
Minimum markup for maximum quality?&Yes&No\\

\bottomrule

\end{tabulary}
\end{minipage}
\end{table}

\end{frame}

\begin{frame}[fragile]

\frametitle{The old way was complicated}
\label{theoldwaywascomplicated}

\begin{verbatim}
<p>In order to create valid 
<a href="http://en.wikipedia.org/wiki/HTML">HTML</a>, you 
need properly coded syntax that can be cumbersome for 
&#8220;non-programmers&#8221; to write. Sometimes, you
just want to easily make certain words <strong>bold
</strong>, and certain words <em>italicized</em> without
having to remember the syntax. Additionally, for example,
creating lists:</p>

<ul>
<li>should be easy</li>
<li>should not involve programming</li>
</ul>
\end{verbatim}

\end{frame}

\mode<all>
\input{mmd-beamer-footer}

\end{document}\mode*

