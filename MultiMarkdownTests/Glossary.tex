\input{mmd-article-header}
\def\mytitle{MultiMarkdown Glossary Test}
\input{mmd-article-begin-doc}
\part{Glossary}
\label{glossary}

MultiMarkdown has a special format for footnotes that should represent
glossary terms. This doesn't make much difference in XHTML (because there is
no such thing as a glossary in XHTML), but can be used to generate a glossary
within LaTeX documents.

For example, let's have an entry for \texttt{glossary}.\newglossaryentry{Glossary }{name={Glossary },description={A section at the end {\ldots}}}\glsadd{Glossary } And what about
ampersands?\newglossaryentry{& }{sort={ampersand},name={\& },description={A punctuation mark {\ldots}}}\glsadd{& }

Since we want the ampersand entry to be sorted with the a's, and not with
symbols, we put in the optional sort key \texttt{ampersand} to control sorting.

\begin{verbatim}
[^glossary]: glossary: Glossary 
    A section at the end ...

[^amp]: glossary: & (ampersand)
    A punctuation mark ...
\end{verbatim}


\input{mmd-memoir-footer}

\end{document}
